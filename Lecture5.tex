\documentclass[pdf,ps,8pt]{beamer}

\usepackage{lmodern}
\usepackage{amsmath}
\usepackage{color}
\usepackage{bbold}
\usepackage{cancel}
\usepackage{slashed}
\usepackage{graphicx}
\usepackage{verbatim}
\usepackage{textcomp}

\usetheme{Singapore}

\definecolor{palegray}{rgb}{0.82,0.822,0.82}
\newcommand{\preliminary}{
{ \rput{30}(7,-4.0){\fontsize{40}{40}\selectfont {\color{palegray}Preliminary Preliminary}} }
}

\newcommand{\textapprox}{\raisebox{0.5ex}{\texttildelow}}

\newcommand{\miniscule}{\fontsize{3pt}{4pt}\selectfont}

\def\MSbar{$\overline{\mathrm{MS}}$}
\def\gev{\,\mathrm{GeV}}
\def\mev{\,\mathrm{MeV}}
\def\fm{\,\mathrm{fm}}
\def\SU{\mathrm{SU}}
\def\su#1#2{\SU(#1)_\mathrm{#2}}
\def\rpisq{\langle r_\pi^2\rangle}
% final values
\def\rpisqsim{0.38(4)}  % pole fit for 330 MeV pion
\def\rpisqsu2sim{0.354(31)}  % su2 fit for 330 MeV pion
\def\rpisqsimlong{0.382(42)}
\def\rpisqphys{0.418(31)} % SU(2) chi extrap
\def\rpisqphyslong{0.418(31)}
\def\lsixr{-0.0093(10)} % SU(2)
\def\Lniner{0.0031(6)}  % SU(3)

\newcommand{\chpt}{\chi^{\rm PT}}
\newcommand{\tchpt}{$\chi^{\rm PT}$}
\newcommand{\tchptthree}{$\chi^{\rm PT}_3$}
\newcommand{\tchpttwo}{$\chi^{\rm PT}_2$}

\newcommand{\xiav}{\langle\,\xi\,\rangle}
\newcommand{\xisqav}{\langle\,\xi^2\,\rangle}


\newcommand{\mD}{\left(\begin{array}{cc} \DO & \Dd \\  \Ddb&\DOb \end{array} \right)}

\newcommand{\Ob}{\bar{\Omega}}

\newcommand{\DO}{D_\Omega}
\newcommand{\Dd}{D_\partial}
\newcommand{\DOi}{D_\Omega^{-1}}
\newcommand{\DOid}{D_\Omega^{-\dagger}}
\newcommand{\Pd} {\mathbb{P}_\partial}
\newcommand{\PO} {\mathbb{P}_\Omega}

\newcommand{\DOb}{D_{\bar{\Omega}}}
\newcommand{\Ddb}{D_{\bar{\partial}}}
\newcommand{\DObi}{D_{\bar{\Omega}}^{-1}}
\newcommand{\DObid}{D_{\bar{\Omega}}^{-\dagger}}
\newcommand{\Pdb}{\mathbb{P}_{\bar{\partial}}}
\newcommand{\POb} {\mathbb{P}_{\bar\Omega}}

\newcommand{\Phidb}{\mathbb{\phi}_{\bar{\partial}}}
\newcommand{\etadb}{\mathbb{\eta}_{\bar{\partial}}}

\newcommand{\hDO}{\hat D_\Omega}
\newcommand{\hDd}{\hat D_\partial}
\newcommand{\hDOi}{\hat D_\Omega^{-1}}
\newcommand{\hPd} {\hat{\mathbb{P}}_\partial}

\newcommand{\hDOb}{\hat D_{\bar{\Omega}}}
\newcommand{\hDdb}{\hat D_{\bar{\partial}}}
\newcommand{\hDObi}{\hat D_{\bar{\Omega}}^{-1}}
\newcommand{\hPdb}{\hat{\mathbb{P}}_{\bar{\partial}}}

\newcommand{\mul}[1]{\left(\begin{array}{cc}#1 & 0 \\ 0& 0\end{array}\right)}
\newcommand{\mur}[1]{\left(\begin{array}{cc}0  & #1\\ 0& 0\end{array}\right)}
\newcommand{\mll}[1]{\left(\begin{array}{cc}0  & 0 \\ #1 & 0\end{array}\right)}
\newcommand{\mlr}[1]{\left(\begin{array}{cc}0  & 0 \\ 0& #1\end{array}\right)}

\newcommand{\mDO}{\mul{ \DO}}
\newcommand{\mDd}{\mur{ \Dd}}
\newcommand{\mDOi}{\mul{\DOi}}
\newcommand{\mPd} {\mlr{\Pd}}

\newcommand{\mDOb}{\mlr{\DOb}}
\newcommand{\mDdb}{\mll{\Ddb}}
\newcommand{\mDObi}{\mlr{\DObi}}
\newcommand{\mPdb}{\mul{\Pdb}}
\newcommand{\rmod}{\mathrm{mod}}
\newcommand{\rdiv}{\mathrm{div}}

\newcommand{\link}[1]{\href{#1}{ {\color{blue} #1} }}

\beamertemplatenavigationsymbolsempty
\begin{document}

\begin{frame}[fragile]\small\frametitle{  Computational Methods (practice) -  Lecture 5    }

  \begin{center}
 
  {\color{red} Peter Boyle} (BNL, Edinburgh)

  \begin{itemize}
  \item Fermion actions 
  \item Sources and hadronic correlation functions
  \item Meson two point functions
  \item Final words
  \end{itemize}

\end{center}  

\end{frame}

\begin{frame}[fragile]\small\frametitle{ Fermion actions }
\end{frame}

\begin{frame}[fragile]\small\frametitle{ Sources and hadronic correlation functions }
\end{frame}

\begin{frame}[fragile]\small\frametitle{ Meson two point functions }
\end{frame}

\begin{frame}[fragile]\small\frametitle{ Meson three point functions }
\end{frame}

\begin{frame}[fragile]\small\frametitle{ Ward identities }
\end{frame}

\begin{frame}[fragile]\small\frametitle{ Final words }
  \begin{itemize}
  \item Aims
  \begin{itemize}
  \item convince you that LQCD software can be elegant, portable and fast
  \item convince you that algorithms can be easy to implement
  \item convince you that code can be elegant, portable and fast
  \item convince you to get your hands dirty!
  \item draw connections between a sample of the core algorithms \& methods of LQCD
  \item keep the exposition simple while still covering the depth
  \end{itemize}
  \item Please provide feedback: what worked and what didn't
  \item I hope you enjoyed the course
  \end{itemize}
\end{frame}


\end{document}



